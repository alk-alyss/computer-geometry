\documentclass{article}

\usepackage[greek]{babel}

\usepackage[a4paper, margin=1.15in]{geometry}

\usepackage{graphicx}

\usepackage{fontspec}
\setmainfont{DejaVuSansMono Nerd Font}
\setsansfont{DejaVuSansMono Nerd Font}

\title{Εργαστήριο 1}
\author{Αλκίνοος Αλυσσανδράκης 1072752}
\date{}

\begin{document}

\maketitle

\section{Περιγραφή αλγορίθμων}

\subsection{Graham Scan}

\begin{enumerate}
	\item Ταξινομούμε τα σημεία του pointcloud με βάση τη γωνία τους σε πολικές
		συντεταγμένες
	\item Βρίσκουμε το σημείο του pointcloud που έχει το μικρότερο Y
	\item Προσθέτουμε το σημείο σε μια στοίβα
	\item Προσθέτουμε το επόμενο σημείο από το pointcloud στη στοίβα
	\item Επανάληψη για όλα τα σημεία στο pointcloud
	\item Παίρνουμε το επόμενο σημείο από το pointcloud και βρίσκουμε τη γωνία που σχηματίζεται ανάμεσα σε αυτό και τα δύο σημεία στην κορυφή της στοίβας
	\item Για όσο η γωνία είναι δεξιόστροφη αφαιρούμε το σημείο που βρίσκεται στην κορυφή της στοίβας
	\item Προσθέτουμε το σημείο που πήραμε από το pointcloud στη στοίβα
	\item Τέλος επανάληψης
	\item Η στοίβα περιέχει τα σημεία του pointcloud που αποτελούν το Convex Hull
\end{enumerate}

\subsection{Quickhull}

\begin{enumerate}
	\item Βρίσκουμε τα σημεία Α, Β του pointcloud που έχουν μικρότερο και μεγαλύτερο X αντίστοιχα
	\item Σχηματίζουμε τα ευθύγραμμα τμήματα ΑΒ και ΒΑ
	\item Καλούμε τη συνάρτηση F δύο φορές με ορίσματα τα σημεί
	\item Συνάρτηση F(S, E, pointcloud)
	\begin{enumerate}
	\item \label{repeat} Με δεδομένα δύο σημεία S, E βρίσκουμε το σημείο P που βρίσκεται αριστερά από το ευθύγραμμο τμήμα SE και απέχει περισσότερο από αυτό
	\item Σχεδιάζουμε τα ευθύγραμμα τμήματα SP και PE και βρίσκουμε το σύνολο των σημείων που βρίσκεται αριστερά από αυτά τα ευθύγραμμα τμήματα
	\item Αν ένα από αυτά τα σύνολα είναι κενά τότε το P ανήκει στο Convex Hull οπότε το αποθηκεύουμε σε μια λίστα
	\item goto \ref{repeat} για τα
\end{enumerate}

\subsection{Jarvis Match}

\section{Τμήμα Convex Hull φανερό από ένα σημείο}

\section{Σύγκριση χρόνου εκτέλεσης αλγορίθμων Convex Hull}

\end{document}
