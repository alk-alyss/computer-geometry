\documentclass{article}

\usepackage[greek]{babel}

\usepackage[a4paper, margin=1.15in]{geometry}

\usepackage{graphicx}

\usepackage{fontspec}
\setmainfont{DejaVuSansMono Nerd Font}
\setsansfont{DejaVuSansMono Nerd Font}

\title{Εργαστήριο 1}
\author{Αλκίνοος Αλυσσανδράκης 1072752}
\date{}

\begin{document}

\maketitle

\section{Περιγραφή αλγορίθμων}

\subsection{Graham Scan}

\begin{enumerate}
	\item Ταξινομούμε τα σημεία του pointcloud με βάση τη γωνία τους σε πολικές
		συντεταγμένες
	\item Βρίσκουμε το σημείο του pointcloud που έχει το μικρότερο Y
	\item Προσθέτουμε το σημείο σε μια στοίβα
	\item Προσθέτουμε το επόμενο σημείο από το pointcloud στη στοίβα
	\item Επανάληψη για όλα τα σημεία στο pointcloud
	\item Παίρνουμε το επόμενο σημείο από το pointcloud και βρίσκουμε τη γωνία που σχηματίζεται ανάμεσα σε αυτό και τα δύο σημεία στην κορυφή της στοίβας
	\item Για όσο η γωνία είναι δεξιόστροφη αφαιρούμε το σημείο που βρίσκεται στην κορυφή της στοίβας
	\item Προσθέτουμε το σημείο που πήραμε από το pointcloud στη στοίβα
	\item Τέλος επανάληψης
	\item Η στοίβα περιέχει τα σημεία του pointcloud που αποτελούν το Convex Hull
\end{enumerate}

\subsection{Quickhull}

\begin{enumerate}
	\item Βρίσκουμε τα σημεία Α, Β του pointcloud που έχουν μικρότερο και μεγαλύτερο X αντίστοιχα
	\item Σχηματίζουμε τα ευθύγραμμα τμήματα ΑΒ και ΒΑ
	\item Καλούμε τη συνάρτηση F δύο φορές με ορίσματα Α,Β,pointcloud και Β,Α,pointcloud αντίστοιχα
	\item Συνάρτηση F(S, E, pointcloud)

	\begin{enumerate}
		\item Αν το pointcloud είναι κενό η συνάρτηση τερματίζει
		\item Με δεδομένα δύο σημεία S, E βρίσκουμε το σημείο P που βρίσκεται αριστερά από το κατευθυνόμενο ευθύγραμμο τμήμα SE και απέχει περισσότερο από αυτό
		\item Το P ανήκει στο Convex Hull οπότε το αποθηκεύουμε σε μια λίστα
		\item Σχεδιάζουμε τα ευθύγραμμα τμήματα SP και PE και βρίσκουμε το σύνολο των σημείων Q που βρίσκεται αριστερά από το SP και το σύνολο των σημείων R που βρίσκεται αριστερά από το PE
		\item Καλούμε αναδρομικά την συνάρτηση F με ορίσματα S,P,Q και μια δεύτερη φορά με ορίσματα P,E,R
	\end{enumerate}

	\item Στο τέλος αυτής της διαδικασίας έχουμε μια λίστα που περιέχει όλα τα σημεία του pointcloud που ανήκουν στο Convex Hull

\end{enumerate}

\subsection{Jarvis Match}

\begin{enumerate}
	\item Βρίσκουμε το σημείο του P pointcloud που έχει το μικρότερο X
	\item Προσθέτουμε το σημείο σε μια λίστα αφού αυτό ανήκει στο Convex Hull
	\item \label{repeat} Διαλέγουμε τυχαία ένα σημείο Q του pointcloud
	\item \label{checkpoint} Σχηματίζουμε το ευθύγραμμο τμήμα PQ και βρίσκουμε το σύνολο R των σημείων που βρίσκονται στα αριστερά αυτού του ευθύγραμμου τμήματος
	\item Αν το σύνολο δεν R είναι κενό διαλέγουμε ένα καινούριο τυχαίο σημείο Q από το σύνολο R και επαναλαμβάνουμε το βήμα \ref{checkpoint}
	\item Αλλιώς το σημείο Q ανήκει στο Convex Hull οπότε το προσθέτουμε στη λίστα
	\item Αν το σημείο Q είναι το σημείο στην αρχή της λίστας σημείων του Convex Hull τότε η διαδικασία τερματίζει και έχουμε βρει ολόκληρο το Convex Hull
	\item Αλλιώς μετονομάζουμε το Q σε P και πηγαίνουμε στο βήμα \ref{repeat}
\end{enumerate}

\section{Τμήμα Convex Hull φανερό από ένα σημείο}

Με δεδομένο ένα σημείο A εκτός του Convex Hull εκτελούμε την εξής διαδικασία για κάθε σημείο B του Convex Hull
\begin{enumerate}
	\item Σχηματίζουμε το διάνυσμα V που ξεκινάει από το σημείο A και καταλήγει στο σημείο B και βρίσκουμε το διάνυσμα U που είναι κάθετο στο V
	\item Βρίσκουμε το σημείο C που βρίσκεται στη θέση V+U
	\item Σχηματίζουμε τα ευθύγραμμα τμήματα AB και BC
	\item Βρίσκουμε το σύνολο R0 των σημείων του Convex Hull που βρίσκεται στα δεξιά του BC
	\item Βρίσκουμε το σύνολο R1 των σημείων του R0 που βρίσκεται στα αριστερά του AB και το σύνολο R2 των σημείων που βρίσκεται στα δεξιά του ΑΒ
	\item Αν το R1 ή το R2 είναι κενό τότε το σημείο Β είναι ορατό από το σημείο Α οπότε το προσθέτουμε στη λίστα των ορατών σημείων
\end{enumerate}

\noindent
Στο τέλος της διαδικασίας έχουμε μια λίστα με σημεία του Convex Hull τα οποία είναι ορατά από το σημείο A

\section{Σύγκριση χρόνου εκτέλεσης αλγορίθμων Convex Hull}

\includegraphics[width=\textwidth]{"algorithm plot.png"}

\noindent
Παρατηρούμε ότι στους αλγόριθμους Graham Scan και Quickhull ο χρόνος εκτέλεσης αυξάνεται γραμμικά με τον αριθμό
των σημείων του pointcloud (με τον αλγόριθμο quickhull να είναι πιο γρήγορος σε σχέση με τον αλγόριθμο
Graham Scan), ενώ ο χρόνος εκτέλεσης του αλγόριθμου Jarvis Match αυξάνεται με μη γραμμικό τρόπο λόγω του ότι
η πολυπλοκότητα αυτού του αλγόριθμου εξαρτάται από την έξοδό του, δηλαδή τον αριθμό των σημείων που ανήκουν
στο Convex Hull

\end{document}
