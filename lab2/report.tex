\documentclass{article}

\usepackage[greek]{babel}

\usepackage[a4paper, margin=1.15in]{geometry}

\usepackage{graphicx}
\usepackage{amsmath}

\usepackage{fontspec}
\setmainfont{DejaVuSansMono Nerd Font}
\setsansfont{DejaVuSansMono Nerd Font}

\title{Εργαστήριο 2}
\author{Αλκίνοος Αλυσσανδράκης 1072752}
\date{}

\begin{document}

\maketitle

\section{Circle 2D}
\subsection{constructor}
Για να φτιάξουμε ένα κύκλο από τρια σημεία v1, v2, v3 βρίσκουμε πρώτα το κέντρο
του κύκλου λύνοντας το σύστημα

$
A =
\begin{bmatrix}
	v2.x - v1.x & v2.y - v2.y\\
	v3.x - v2. & v3.y - v2.y
\end{bmatrix}
$

$
b =
\begin{bmatrix}
	|| v2 ||  - || v1 || \\
	|| v3 ||  -  || v2 ||
\end{bmatrix}
$

$ Ax = b$

\noindent
όπου x το κέντρο του κύκλου. Ύστερα βρίσκουμε την ακτίνα η οποία είναι η απόσταση
ενός των σημείων από το κέντρο

\subsection{contains}
Ένα σημείο ανήκει στον κύκλο αν η απόστασή του από το κέντρο είναι μικρότερη ή ίση της ακτίνας

\subsection{o3d\_lineset}
Για να εμφανίσουμε τον κύκλο στην οθόνη πρέπει να τον μετατρέψουμε σε πολλά μικρά ευθύγραμμα
τμήματα. Παίρνουμε πολλά σημεία πάνω στον κύκλο τα οποία ισαπέχουν σε γωνία από το κέντρο
και σχηματίζουμε τα ευθύγραμμα τμήματα ανάμεσα σε αυτά τα σημεία. Αυτά τα ευθύγραμμα
τμήματα είναι που τελικά εμφανίζονται στην οθόνη

\section{Triangle2D}
\subsection{contains}
Τα σημεία του τριγώνου είναι τα v1, v2, v3
Το σημείο που ψάχνουμε αν είναι εντός του τριγώνου είναι το v.
Βρίσκουμε αν το v είναι εντός του τριγώνου με την εξής μέθοδο:

\begin{itemize}
	\item Σχηματίζουμε τα τρίγωνα (v1, v2, v), (v1, v3, v), (v2, v3, v)
	\item Αν το άθροισμα των εμβαδών αυτών των τριών τριγώνων ισούται με το εμβαδόν του αρχικού τριγώνου τότε το v βρίσκεται εντός του τριγώνου
	\item Αλλιώς βρίσκεται εκτός του τριγώνου
\end{itemize}

\subsection{has\_vertex}
Αν ένα σημείο v απέχει από μια από τις κορυφές του τριγώνου λιγότερο από $10^-8$
τότε θεωρούμε ότι το v είναι μια από τις κορυφές του τριγώνου

\subsection{has\_common\_edge}
Για δύο τρίγωνα αν αυτά έχουν δύο κοινές κορυφές τότε αυτα έχουν μια κοινή πλευρά

\section{Gui}
\subsection{split}
Δοθέντος ενός τριγώνου με κορυφές v1, v2, v3 και ένα σημείο v εντός του τριγώνου
σχηματίζουμε τα τρίγωνα (v1, v2, v), (v1, v3, v), (v2, v3, v). Προσθέτουμε τα τρίγωνα
αυτά στις γεωμετρίες που εμφανίζονται στην οθόνη και αφαιρούμε το αρχικό τρίγωνο

\subsection{circumcircle}
Για να βρούμε το περιγεγραμμένο κύκλο ενός τριγώνου, παίρνουμε τα τρία σημεία του τριγώνου v1, v2, v3 και κατασκευάζουμε ένα αντικείμενο Circle2D με αυτά τα σημεία οπότε από
τον constructor αυτού του αντικειμένου δημιουργείται ο κύκλος. Καλώντας στη συνέχεια τη
μέθοδο as\_o3d\_lineset αυτού του αντικειμένου παίρνουμε τα ευθύγραμμα τμήματα που θα
εμφανίσουμε στην οθόνη για να αναπαραστήσουμε τον περιγεγραμμένο κύκλο

\subsection{delauney violations}
Βρίσκουμε τα delauney violations αν για ένα τρίγωνο βρούμε πρώτα τον περιγεγραμμένο
κύκλο του και στη συνέχεια για κάθε γειτονικό τρίγωνο ελέγξουμε αν κάποια από τις κορυφές
του βρίσκεται εντός του κύκλου. Αν ναι τότε αυτή η κορυφή αποτελεί delauney violation

\subsection{flip}
Για να εκτελέσουμε μια διαδικασία flip σε δύο τρίγωνα βρίσκουμε πρώτα τις κοινές και τις
μη κοινές κορυφές των δύο τριγώνων. Ύστερα σχηματίζουμε δύο νέα τρίγωνα που έχουν ως
κορυφές τα δύο μη κοινά σημεία των αρχικών τριγώνων και ένα από τα δύο κοινά σημεία.
Προσθέτουμε τα τρίγωνα αυτά στις γεωμετρίες που εμφανίζουμε στην οθόνη και αφαιρούμε τα
αρχικά τρίγωνα

\section{Overlapping Triangles}
Με δεδομένα τα σημεία από δύο τρίγωνα βρίσκουμε την περιοχή επικάλυψης αυτών των δύο με
την εξής μέθοδο:

\begin{itemize}
	\item Από τα σημεία των δύο τριγώνων κατασκευάζουμε τα ευθύγραμμα τμήματα
	\item Βρίσκουμε τα σημεία στα οποία αυτά τα ευθύγραμμα τμήματα τέμνονται
	\item Βρίσκουμε επίσης και τις κορυφές που περιέχονται μέσα σε κάποιο
		τρίγωνο
	\item Η περιοχή επικάλυψης είναι ο χώρος που περικλύεται από τα σημεία τομής και τις
		κορυφές που βρίσκονται εντός τριγώνων
\end{itemize}

\end{document}
