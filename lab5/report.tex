\documentclass{article}

\usepackage[greek]{babel}

\usepackage[a4paper, margin=1.15in]{geometry}

\usepackage{graphicx}
\usepackage{amsmath}

\usepackage{fontspec}
\setmainfont{DejaVuSansM Nerd Font}
\setsansfont{DejaVuSansM Nerd Font}

\title{Εργαστήριο 5}
\author{Αλκίνοος Αλυσσανδράκης 1072752}
\date{}

\begin{document}

\maketitle

\section{Εισαγωγή}
Έστω ένα τρισδιάστατο mesh που αποτελείται από ένα σύνολο κορυφών V που αποτελείται από
n διανύσματα και ένα σύνολο ακμών E που αποτελείται απο ζεύγη κορυφών του συνόλου V.
Τα δεδομένα που είναι αποθηκευμένα για αυτό το mesh είναι ο πίνακας X που περιέχει όλες
τις κορυφές και ο πίνακας T που περιέχει όλα τα τρίγωνα του mesh.
Θέλουμε σε αυτό το mesh να εφαρμόσουμε κάποιες τεχνικές για την εξομάλυνση του. Η μια
τεχνική είναι το Laplacian Smoothing και η άλλη είναι το Taubin Smoothing

\section{Laplacian Smoothing}
Για να εφαρμόσουμε Laplacian Smoothing στο mesh με n κορυφές πρεπεί πρώτα να σχηματίσουμε
τους εξής πίνακες:

\begin{itemize}
	\item $A$: ο τετραγωνικός πίνακας συνδεσιμότητας του mesh μεγέθους n
		όπου $A(i,j) = 1$ όταν η κορυφή i συνδέεται με την κορυφή j
	\item $D$: διαγώνιος πίνακας όπου \[D(i,i) = \sum_{j=0}^{n} A(i, j)\]
	\item $L$: ο πίνακας που προκύπτει από την πράξη $L = I - D^{-1}A$
\end{itemize}

\noindent
Το Laplacian Smooting θα εφαρμοστεί για κάθε κορυφή χρησιμοποιώντας τον τύπο
\[p_i^{t+1} = p_i^{t} - \lambda Lp_i^{t}\]
όπου $p_i^{t}$ είναι η ι-οστή κορυφή του mesh πριν την εξομάλυνση, $p_i^{t+1}$ η ίδια
κορυφή μετά την εξομάλυνση και $\lambda \in (0, 1)$ ο συντελεστής εξομάλυνσης.
Η πράξη $Lp_i^{t}$ υπολογίζει τις διαφορικές συντεταγμένες της κορυφής i σε σχέση με το
κέντρο βάρους των γειτονικών της κορυφών.

Έτσι εφαρμόζοντας τον παραπάνω τύπο πολλές φορές μπορούμε να πετύχουμε όλο και μεγαλύτερα
επίπεδα εξομάλυνσης, ενώ όσο το $\lambda$ πλησιάζει στο 1 τόσο πιο μεγάλη είναι η
εξομάλυνση σε κάθε επανάληψη.


\section{Taubin Smooting}


\end{document}
