\documentclass{article}

\usepackage[greek]{babel}

\usepackage[a4paper, margin=1.15in]{geometry}

\usepackage{graphicx}
\usepackage{amsmath}
\usepackage{subcaption}

\usepackage{fontspec}
\setmainfont{DejaVuSansM Nerd Font}
\setsansfont{DejaVuSansM Nerd Font}

\title{Εργαστήριο 5}
\author{Αλκίνοος Αλυσσανδράκης 1072752}
\date{}

\begin{document}

\maketitle

\section{Εισαγωγή}
Έστω ένα τρισδιάστατο mesh που αποτελείται από ένα σύνολο κορυφών V που αποτελείται από
n διανύσματα και ένα σύνολο ακμών E που αποτελείται απο ζεύγη κορυφών του συνόλου V.
Τα δεδομένα που είναι αποθηκευμένα για αυτό το mesh είναι ο πίνακας X που περιέχει όλες
τις κορυφές και ο πίνακας T που περιέχει όλα τα τρίγωνα του mesh.
Θέλουμε σε αυτό το mesh να εφαρμόσουμε κάποιες τεχνικές για την εξομάλυνση του. Η μια
τεχνική είναι το Laplacian Smoothing και η άλλη είναι το Taubin Smoothing

\section{Laplacian Smoothing}
Για να εφαρμόσουμε Laplacian Smoothing στο mesh με n κορυφές πρεπεί πρώτα να σχηματίσουμε
τους εξής πίνακες:

\begin{itemize}
	\item $A$: ο τετραγωνικός πίνακας συνδεσιμότητας του mesh μεγέθους n
		όπου $A(i,j) = 1$ όταν η κορυφή i συνδέεται με την κορυφή j
	\item $D$: διαγώνιος πίνακας όπου \[D(i,i) = \sum_{j=0}^{n} A(i, j)\]
	\item $L$: ο πίνακας που προκύπτει από την πράξη $L = I - D^{-1}A$
\end{itemize}

\noindent
Το Laplacian Smoothing θα εφαρμοστεί για κάθε κορυφή χρησιμοποιώντας τον τύπο
\[p_i^{t+1} = p_i^{t} - \lambda Lp_i^{t}\]
όπου $p_i^{t}$ είναι η ι-οστή κορυφή του mesh πριν την εξομάλυνση, $p_i^{t+1}$ η ίδια
κορυφή μετά την εξομάλυνση και $\lambda \in (0, 1)$ ο συντελεστής εξομάλυνσης.
Η πράξη $Lp_i^{t}$ υπολογίζει τις διαφορικές συντεταγμένες της κορυφής i σε σχέση με το
κέντρο βάρους των γειτονικών της κορυφών.

Έτσι εφαρμόζοντας τον παραπάνω τύπο πολλές φορές μπορούμε να πετύχουμε όλο και μεγαλύτερα
επίπεδα εξομάλυνσης, ενώ όσο το $\lambda$ πλησιάζει στο 1 τόσο πιο μεγάλη είναι η
εξομάλυνση σε κάθε επανάληψη.

\section{Taubin Smoothing}
Η διαδικασία του Laplacian Smoothing όμως έχει ένα πρόβλημα. Ο τρόπος με τον οποίο
λειτουργεί αυτού του είδους η εξομάλυνση προκαλεί μια συρρίκνωση του μοντέλου σε
κάθε επανάληψη. Έτσι μετά από πολλές επαναλήψεις το μοντέλο καταλήγει να χάνει
τον όγκο του και εν τέλει να είναι ομαλό, αλλά όχι όμοιο με το αρχικό.

Το πρόβλημα αυτό έρχεται να λύσει η διαδικασία του Taubin smoothing η οποία λειτουργεί
με παρόμοιο τρόπο όπως το Laplacian Smoothing. Έχουμε τα ίδια δεδομένα και υπολογίζουμε
τους ίδιους πίνακες όμως όταν έρθει η ώρα να κάνουμε τις επαναλήψεις εξομάλυνσης
υπάρχει μια μικρή διαφορά.

Σε κάθε επανάληψη εφαρμόζουμε τον ίδιο τύπο όπως και στο Laplacian Smoothing
\[p_i^{t+1} = p_i^{t} - \lambda Lp_i^{t}\]
στη συνέχεια όμως εφαρμόζουμε και τον εξής τύπο:
\[p_i^{t+1} = p_i^{t} + \mu Lp_i^{t}\]
ο τύπος αυτός είναι πολύ παρόμοιος με τον αρχικό με τη διαφορά ότι αντί να αφαιρεί την
τιμή $\lambda Lp_i^{t}$, προσθέτει την τιμή $\mu Lp_i^{t}$, δηλαδή αντί να συρρικνώνει
το μοντέλο το μεγεθύνει. Έτσι ο πρώτος τύπος εξομαλύνει το μοντέλο με συρρίκνωση, ο
δεύτερος το εξομαλύνει με μεγέθυση, άρα ο συνδυασμός τους καταφέρνει εξομάλυνση του
μοντέλου χωρίς αλλαγή στον όγκο του. Οι τιμές $\lambda \in (0, 1)$ και $\mu \in (0, 1)$
είναι οι συντελεστές της εξομάλυνσης και καθορίζουν τον βαθμό επιρροής του πρώτου και
του δεύτερου τύπου στην διαδικασία.

\section{Παραδείγματα}

\begin{figure}[h]
	\center
	\includegraphics[width=0.5\textwidth]{"1.png"}
	\caption{Αρχικό μοντέλο}
\end{figure}
\begin{figure}[h]
	\begin{subfigure}{0.5\textwidth}
		\includegraphics[width=0.9\linewidth]{"2.png"}
		\caption{Laplacian Smoothing, $\lambda = 0.5$}
	\end{subfigure}
	\begin{subfigure}{0.5\textwidth}
		\includegraphics[width=0.9\linewidth]{"3.png"}
		\caption{Taubin smooting, $\lambda = 0.5, \mu = 0.5$}
	\end{subfigure}
	\caption{10 επαναλήψεις}
\end{figure}

\end{document}
